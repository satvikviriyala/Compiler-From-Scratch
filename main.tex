\documentclass{beamer}
\usetheme{Warsaw}
\usepackage{amsmath}
\usepackage{graphicx}
\usepackage{tikz}
\usepackage{listings}
\usepackage{xcolor}

\title{Mathematical Typesetting in LaTeX}
\subtitle{A Comprehensive Guide}
\author{Your Name}
\institute{Your Institution}
\date{February 4, 2025}

\lstset{
    basicstyle=\ttfamily\small,
    keywordstyle=\color{blue},
    commentstyle=\color{green!60!black},
    numbers=left,
    numberstyle=\tiny\color{gray}
}

\begin{document}

\begin{frame}
\titlepage
\end{frame}

\section{Introduction}
\begin{frame}{What is Mathematical Typesetting?}
\begin{itemize}
\item Professional presentation of mathematical notation
\item Essential for:
\begin{itemize}
\item Research papers (e.g., \cite{einstein})
\item Technical documentation
\item Academic textbooks
\end{itemize}
\item LaTeX advantages:
\begin{itemize}
\item Precise control over layout
\item Extensive symbol library
\item Automatic numbering
\end{itemize}
\end{itemize}
\end{frame}

\section{Basic Elements}
\begin{frame}[fragile]{Inline vs Display Math}
\begin{columns}
\column{0.5\textwidth}
\begin{lstlisting}
Einstein's equation: $E=mc^2$

Display equation:
\[
\sum_{n=1}^\infty \frac{1}{n^2} 
= \frac{\pi^2}{6}
\]
\end{lstlisting}

\column{0.5\textwidth}
Output:\\
Einstein's equation: \( E=mc^2 \)

Display equation:
\[
\sum_{n=1}^\infty \frac{1}{n^2} = \frac{\pi^2}{6}
\]
\end{columns}
\end{frame}

\begin{frame}[fragile]{Fractions and Roots}
\begin{lstlisting}
\[
\sqrt[3]{\frac{x^2}{y^3}} \quad
\frac{\partial^2 f}{\partial x^2}
\]
\end{lstlisting}

Output:
\[
\sqrt[3]{\frac{x^2}{y^3}} \quad \frac{\partial^2 f}{\partial x^2}
\]

\begin{block}{Continued Fractions}
\begin{lstlisting}
\cfrac{1}{1+\cfrac{2}{1+\cfrac{3}{1+\dots}}}
\end{lstlisting}
\end{block}
\end{frame}

\section{Advanced Features}
\begin{frame}[fragile]{Matrices and Arrays}
\begin{columns}
\column{0.5\textwidth}
\begin{lstlisting}
\[
\begin{pmatrix}
1 & 0 \\
0 & 1
\end{pmatrix}
\]
\end{lstlisting}

\column{0.5\textwidth}
Output:
\[
\begin{pmatrix}
1 & 0 \\
0 & 1
\end{pmatrix}
\]
\end{columns}

\begin{alertblock}{Matrix Variants}
\begin{itemize}
\item \texttt{bmatrix}: Square brackets
\item \texttt{vmatrix}: Vertical bars
\item \texttt{Bmatrix}: Curly braces
\end{itemize}
\end{alertblock}
\end{frame}

\begin{frame}[fragile]{Multi-line Equations}
\begin{lstlisting}
\begin{align}
f(x) &= (a+b)^3 \nonumber \\
     &= a^3 + 3a^2b \nonumber \\
     &\quad + 3ab^2 + b^3
\end{align}
\end{lstlisting}

Output:
\begin{align}
f(x) &= (a+b)^3 \nonumber \\
     &= a^3 + 3a^2b \nonumber \\
     &\quad + 3ab^2 + b^3
\end{align}
\end{frame}

\section{Symbols and Operators}
\begin{frame}{Common Mathematical Symbols}
\begin{table}
\centering
\begin{tabular}{ll}
\hline
Symbol & Code \\
\hline
\( \forall \) & \texttt{\textbackslash forall} \\
\( \exists \) & \texttt{\textbackslash exists} \\
\( \in \) & \texttt{\textbackslash in} \\
\( \subset \) & \texttt{\textbackslash subset} \\
\( \cup \) & \texttt{\textbackslash cup} \\
\( \cap \) & \texttt{\textbackslash cap} \\
\hline
\end{tabular}
\end{table}
\end{frame}

\begin{frame}{Operator Examples}
\begin{exampleblock}{Common Operators}
\begin{itemize}
\item Summation: \texttt{\textbackslash sum}
\item Product: \texttt{\textbackslash prod}
\item Integral: \texttt{\textbackslash int}
\item Limit: \texttt{\textbackslash lim}
\end{itemize}
\end{exampleblock}

\begin{exampleblock}{Usage}
\begin{lstlisting}
\lim_{x \to 0} \frac{\sin x}{x} = 1
\end{lstlisting}
Output:
\[
\lim_{x \to 0} \frac{\sin x}{x} = 1
\]
\end{exampleblock}
\end{frame}

\section{Best Practices}
\begin{frame}{Equation Numbering}
\begin{itemize}
\item Use \texttt{equation} for single equations
\item Use \texttt{align} for multi-line equations
\item Reference equations with \texttt{\textbackslash label}
\item Maintain consistent numbering style
\item Group related equations
\end{itemize}
\end{frame}

\begin{frame}{Common Errors and Solutions}
\begin{alertblock}{Frequent Mistakes}
\begin{itemize}
\item Missing \$ symbols
\item Unclosed math environments
\item Incorrect package loading
\item Math mode conflicts
\end{itemize}
\end{alertblock}

\begin{block}{Troubleshooting Tips}
\begin{itemize}
\item Check compilation logs
\item Use minimal working examples
\item Consult package documentation
\end{itemize}
\end{block}
\end{frame}

\section{Conclusion}
\begin{frame}{Summary}
\begin{itemize}
\item LaTeX provides superior math typesetting
\item Specialized environments for equations
\item Extensive symbol library
\item Advanced formatting control
\item Professional results
\end{itemize}
\end{frame}

\begin{frame}{Resources}
\begin{itemize}
\item Overleaf Documentation: \url{https://overleaf.com/learn}
\item LaTeX Mathematics: \url{https://en.wikibooks.org/wiki/LaTeX/Mathematics}
\item Comprehensive Symbols List: \url{https://oeis.org/wiki/List_of_LaTeX_mathematical_symbols}
\end{itemize}
\end{frame}

\end{document}
